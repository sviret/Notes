\section{Introduction}

\noindent Tracking is a two-stage process: pattern recognition and track fit. Both steps are performed routinely within CMS, using very performant software algorithms. These algorithms are currently performed during the High-Level Trigger (HLT) stage. The Level 1 (L1) stage is currently not relying on the tracking detectors. This is mainly due to the very short latency available to perform the tracking ($3~\mu s$), but also to the impossibility to extract the tracker data at $40~MHz$.

\noindent LHC luminosity increases will put very strong requirements on CMS L1 processing. In order to keep the current L1 performance at $5\cdot~10^{34}~cm^{-2}s^{-1}$ instantaneous luminosity, L1 capabilities will have to be significantly upgraded. L1 accept rate and total latency will be increased, and tracking information will be added.

\noindent Tracking at L1 means hardware tracking. Hardware track trigger systems have already been developped and successfully exploited at collider experiments (CDF, ZEUS,...). ATLAS plans to install it during upgrade phase I. However, none of these systems have ever worked under CMS L1 latency and rate constraints.  

\noindent The goal of this document is to present the track triggering system foreseen for CMS L1 at LHC phase II. A general description of the system is provided in Section~\ref{sec:Description}. Then, the physics motivation for adding tracking to the L1 chain is presented in Section~\ref{sec:TTI}. Part~\ref{sec:Hardware} provides a detailed description of the different parts of the track trigger infrastructure, along with some preliminary simulations results. Finally, integration, cost and planning consideration are provided in Section~\ref{sec:Planning} 

\clearpage
