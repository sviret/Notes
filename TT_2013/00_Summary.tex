\section{Executive summary}

\noindent The physics reach of CMS for the HL-LHC physics program will depend critically on the capability of using tracking information at L1. A powerful tracking trigger is not something that "would be nice to have" but an absolute necessity for the success of Phase 2 upgrades. This implies that the new CMS tracker must be designed in such a way to allow the implementation of an effective tracking trigger. It will take many years to build the new tracker for Phase 2 and, therefore, its design has to be finalized in a few years from now, most likely by 2016. On the other hand, the tracker design cannot be finalized without a demonstration of the feasibility of an effective tracking trigger. In fact we must realize that a silicon-based tracking trigger at L1 has never been done before on this scale. A silicon-based tracking trigger was successfully used as part of the L2 trigger in CDF (SVT) and is being implemented now at Atlas, also for L2 (FTK), both using the Associative Memory approach. However, extending that technology to the HL-LHC environment and from L2 to L1, is going to require some major R\&D. Major challenges to be tackled are the much higher occupancy and event rate, and the requirement of a much shorter latency. Therefore it becomes crucial, for the CMS Phase 2 upgrades, to demonstrate the technical feasibility of a tracking trigger at L1 on a time scale of a few years from now. 

\noindent Over the past few years, motivated by the challenges described above, CMS carried on a focused R\&D program to advance the state-of-the-art of the technologies for hardware-based pattern recognition and track reconstruction. Specifically, we have been addressing the most important challenges for a tracking trigger by developing a full-mesh ATCA based Data Formatting system, higher density Associative Memory chips, and new algorithms for hardware based track fitting. The long-term goal of this R\&D efforts is to develop all necessary critical technologies to the point where we can ultimately propose them as a viable solution to the CMS L1 tracking trigger problem for HL-LHC. Given the progress made by this R\&D program in the last few years, we believe it is now time to move to the next important step and build a Vertical Slice Demonstration System using high luminosity simulated data to implement and study a "vertical slice" of the full tracking trigger path, measure trigger latency and efficiency, study the overall performance, identify possible bottlenecks and issues and, hopefully, find appropriate solutions. 

\noindent The architecture we have recently proposed for the CMS L1 tracking trigger system is based on ATCA with full-mesh backplane. The large inter-board communication bandwidth provided by the full-mesh backplane is used to time multiplex the high volume (~ 50 Tbps) of incoming data in such a way that the I/O bandwidth demands are manageable at the board and chip level, making it possible for an early technical demonstration with existing technology. The resulting architecture is scalable, flexible and open. For example, it allows different pattern recognition architectures and algorithms to be explored and compared within the same platform. Also, given that AMC specifications are designed to work with both ATCA and MicroTCA, this architecture allows a natural long term integration of TK-DAQ (AMC card based) and TK-TRIG (ATCA based).

\noindent The proposed architecture and system demonstration concept has been well received within the tracker Phase 2 upgrade community and we are now working on the details to better define the concept.  At the same time, we are developing collaborations with international partners within CMS. One of the main activities in the coming year (FY14) will consist of extensive simulation efforts, by physicists, to establish technical specifications based on Phase 2 physics goals. At the same time, students and postdocs from all groups will be offered a unique opportunity to develop hardware experience by getting involved with the design, construction and commissioning of the vertical slice demonstration over the next few years. Some of the groups, for the longer term, are also interested in getting involved with the development of the algorithms for the post-track-finding stages and of the interfaces with the global trigger. 



\clearpage
