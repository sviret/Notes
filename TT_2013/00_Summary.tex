\begin{center}
{\Large\bf Executive summary}
\end{center}

\vspace{0.5cm}

\noindent A powerful tracking trigger is required for the success of the CMS physics program in the HL-LHC era.  Consequently, the design of the Phase-II CMS Tracker must allow for an effective implementation of the tracking trigger.  Although the construction of the Phase-II Tracker will take many years, the system's design must be finalized soon.  A silicon-based L1 tracking trigger has never been realized at this scale and thus it is imperative that the feasibility of the trigger be demonstrated before the design of the Phase-II Tracker is finalized.  A silicon-based track trigger was successfully used as part of the L2 CDF trigger (SVT) and is presently being implemented at L2 in ATLAS (FTK).  Both implementations employed an associative memory (AM) approach.  The experiences of these projects will serve as useful inputs in the design of the CMS L1 tracking trigger, however the higher occupancies anticipated in HL-LHC operation and the low latencies required at L1 present us with a unique set of challenges.

\noindent Motivated by these challenges, CMS has carried out a focused R\&D program to advance the state-of-the-art in hardware-based pattern recognition and track reconstruction.  We have attempted to address the issues of occupancy and latency by developing a ``full-mesh'' ATCA data formatting system, higher density AM chips and new algorithms for hardware-based track finding.  The long-term goal of this R\&D effort is to develop these critical technologies to the point where we can ultimately propose them as a viable solution to the problems of HL-LHC L1 track triggering.  Given the progress made by the R\&D program in the last few years, we believe it is now time to take the important next step of establishing a Vertical Slice Demonstration System.  This system will comprise a full tracking trigger path and will be used with simulated high-luminosity data to measure trigger latency and efficiency, to study overall system performance, and to identify potential bottlenecks and appropriate solutions.

\noindent The full-mesh ATCA architecture we have proposed for the CMS L1 tracking trigger permits high bandwidth inter-board communication.  The full-mesh backplane is used to time-multiplex the high volume of incoming data (~50Tbps) in such a way that I/O demands are manageable at the board and chip level.  The resulting architecture is scalable, flexible, and open and it enables us to undertake an early technical demonstration using existing technology.  The ATCA architecture will allow us to explore and compare various pattern recognition architectures and algorithms within the same platform.  Given that Advanced Mezzanine Card (AMC) specifications are designed to work with both ATCA and microTCA, the architecture naturally allows for the long-term integration of Tracker DAQ (AMC based) and tracking trigger activities.  

\noindent The proposed architecture and system demonstration has been well received in the Phase-II Tracker Upgrade community and we are now working to better define the concepts.  In this document we will describe the tracking trigger architecture and system demonstration as a work plan we propose for CMS Phase-II R\&D.  This {\itshape living document} is intended to define the tracking trigger project and to organize efforts leading to the Technical Proposal, the Vertical Slice Demonstration System and the Technical Design Report.



\clearpage
